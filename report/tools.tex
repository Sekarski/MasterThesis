\chapter{Tools}
\label{chap:tools}
%Python,R - Quels packages, libraries, etc.
\Gls{python} will be our main tool for analysis, but we will also consider using \Gls{r}, and compare, where appropriate, the results using \Gls{python} versus using \Gls{r}, to see if there are any differences in implementation that influence the outcome of our analysis.

\section{\Gls{python}}
\Gls{python} is a multi-purpose high level programming language, which is gaining a lot of traction in the \acrlong{ml} community, for its simple syntax and library support.
The main package we will be using for \acrlong{ml} is 'sklearn' \footnote{\url{https://scikit-learn.org} (June 24, 2021)}.
The main pacakge we will be using for \acrlong{ts} is 'statsmodels' \footnote{\url{https://www.statsmodels.org} (June 24, 2021)}.
For \acrshort{garch} we will use the 'arch' package \footnote{\url{https://arch.readthedocs.io} (June 24, 2021)}.
For data manipulation we will be using 'pandas' \footnote{\url{https://pandas.pydata.org} (June 24, 2021)} and 'numpy' \footnote{\url{https://numpy.org/} (June 24, 2021)}.

\section{\Gls{r}}
\Gls{r} is a programming language designed specifically for statistical computing and graphics and has been the go-to programming language for data scientist in the past. It has a lot of library support as well.


The main package used is 'tidyverse' \footnote{\url{https://www.tidyverse.org} (June 24, 2021)}. This is actually a collection of packages, that all share the same underlying structure, and that are data science orientated.


For \acrshort{garch} modeling we will be using the 'fGarch' package \footnote{\url{https://cran.r-project.org/web/packages/fGarch/index.html} (June 24, 2021)}.