\chapter{Tools}
%Python,R - Quels packages, libraries, etc.
Python will be our main tool for analysis, but we will also consider R, and compare, where appropriate, the results using Python versus using R, to see if there are any large or small differences in implementation that influence the outcome of our analysis.

\section{Python}
Python is a multi-purpose high level programming language, which is gaining a lot of traction in the machine learning community, for its simple syntax and library support.
The main package we will be using for \acrlong{ml} is 'sklearn' (\url{https://scikit-learn.org/}).
The main pacakge we will be using for \acrlong{ts} is 'statsmodels' (\url{https://www.statsmodels.org/}).
For \acrshort{garch} we will use the 'arch' package (\url{https://arch.readthedocs.io/}).
For data manipulation we will be using 'pandas' (\url{https://pandas.pydata.org/} and 'numpy' (\url{https://numpy.org/})

\section{R}
R is a programming language designed specifically for statistical computing and graphics and has been the go-to programming language for data scientist in the past. It has a lot of library support as well.


The main package used is 'tidyverse' (\url{https://www.tidyverse.org/}). This is actually a collection of packages, that all share the same underlying structure, and that are data science orientated.


For \acrshort{garch} modeling we will be using the 'fGarch' package (\url{https://cran.r-project.org/web/packages/fGarch/index.html})